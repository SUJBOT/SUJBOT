% SUJBOT2 Pipeline Prezentace
% Kompilace: xelatex presentation.tex (nebo pdflatex)
\documentclass[aspectratio=169, 11pt]{beamer}

% ============================================================================
% BALÍČKY
% ============================================================================
\usepackage[czech]{babel}
\usepackage[T1]{fontenc}
\usepackage[utf8]{inputenc}
\usepackage{tikz}
\usepackage{fontawesome5}
\usepackage{xcolor}

% ============================================================================
% BARVY
% ============================================================================
\definecolor{activeblue}{HTML}{2563eb}
\definecolor{inactivegray}{HTML}{9ca3af}
\definecolor{textdark}{HTML}{1f2937}
\definecolor{lightgray}{HTML}{f3f4f6}

% ============================================================================
% TÉMA - Minimalistické
% ============================================================================
\usetheme{default}
\usecolortheme{default}

% Odstranění navigačních symbolů
\setbeamertemplate{navigation symbols}{}

% Barvy
\setbeamercolor{normal text}{fg=textdark}
\setbeamercolor{frametitle}{fg=textdark}
\setbeamercolor{title}{fg=textdark}
\setbeamercolor{itemize item}{fg=activeblue}
\setbeamercolor{itemize subitem}{fg=activeblue}

% Fonty
\setbeamerfont{frametitle}{size=\Large, series=\bfseries}
\setbeamerfont{title}{size=\huge, series=\bfseries}

% ============================================================================
% PIPELINE VIZUALIZACE
% ============================================================================
% Použití: \setpipelinestep{1} až \setpipelinestep{5}
% 0 = žádný krok aktivní (titulní slide, závěr)

\newcommand{\currentpipelinestep}{0}
\newcommand{\setpipelinestep}[1]{\renewcommand{\currentpipelinestep}{#1}}

\newcommand{\drawpipeline}{%
  \begin{tikzpicture}[
    step/.style={
      rectangle,
      rounded corners=3pt,
      minimum height=0.6cm,
      minimum width=2cm,
      font=\scriptsize\sffamily,
      align=center
    },
    activestep/.style={
      step,
      fill=activeblue,
      text=white,
      font=\scriptsize\sffamily\bfseries
    },
    inactivestep/.style={
      step,
      fill=lightgray,
      text=inactivegray
    },
    arrow/.style={
      ->,
      >=stealth,
      color=inactivegray,
      thick
    }
  ]
    % Pozice kroků
    \def\stepwidth{2.4cm}

    % Krok 1: Extrakce
    \ifnum\currentpipelinestep=1
      \node[activestep] (s1) at (0,0) {Extrakce};
    \else
      \node[inactivestep] (s1) at (0,0) {Extrakce};
    \fi

    % Krok 2: Chunking
    \ifnum\currentpipelinestep=2
      \node[activestep] (s2) at (\stepwidth,0) {Chunking};
    \else
      \node[inactivestep] (s2) at (\stepwidth,0) {Chunking};
    \fi

    % Krok 3: Knowledge Graph
    \ifnum\currentpipelinestep=3
      \node[activestep] (s3) at (2*\stepwidth,0) {Knowledge Graph};
    \else
      \node[inactivestep] (s3) at (2*\stepwidth,0) {Knowledge Graph};
    \fi

    % Krok 4: Chat
    \ifnum\currentpipelinestep=4
      \node[activestep] (s4) at (3*\stepwidth,0) {Multi-agent};
    \else
      \node[inactivestep] (s4) at (3*\stepwidth,0) {Multi-agent};
    \fi

    % Krok 5: Tools
    \ifnum\currentpipelinestep=5
      \node[activestep] (s5) at (4*\stepwidth,0) {Tools};
    \else
      \node[inactivestep] (s5) at (4*\stepwidth,0) {Tools};
    \fi

    % Šipky mezi kroky
    \draw[arrow] (s1) -- (s2);
    \draw[arrow] (s2) -- (s3);
    \draw[arrow] (s3) -- (s4);
    \draw[arrow] (s4) -- (s5);
  \end{tikzpicture}%
}

% Header template s pipeline
\setbeamertemplate{headline}{%
  \ifnum\currentpipelinestep>0
    \vskip10pt
    \hfill\drawpipeline\hfill\null
    \vskip5pt
    \begin{beamercolorbox}[wd=\paperwidth, ht=0.5pt]{separation line}
      \color{lightgray}\rule{\paperwidth}{0.5pt}
    \end{beamercolorbox}
  \fi
}

% Footline s číslem slidu
\setbeamertemplate{footline}{%
  \hfill%
  \usebeamercolor[fg]{page number in head/foot}%
  \usebeamerfont{page number in head/foot}%
  \insertframenumber\,/\,\inserttotalframenumber%
  \hspace*{10pt}\vskip10pt%
}

% ============================================================================
% METADATA
% ============================================================================
\title{SUJBOT2}
\subtitle{RAG systém pro právní a technické dokumenty}
\author{Tým A}
\date{\today}
\institute{Projekt ADS}

% ============================================================================
% DOKUMENT
% ============================================================================
\begin{document}

% ----------------------------------------------------------------------------
% TITULNÍ SLIDE
% ----------------------------------------------------------------------------
\setpipelinestep{0}
\begin{frame}[plain]
  \vfill
  \centering
  {\usebeamerfont{title}\usebeamercolor[fg]{title}\inserttitle}

  \vskip0.5cm
  {\large\color{inactivegray}\insertsubtitle}

  \vskip1.5cm

  % Mini pipeline přehled
  \begin{tikzpicture}[
    minibox/.style={
      rectangle,
      rounded corners=2pt,
      minimum height=0.5cm,
      minimum width=1.8cm,
      fill=lightgray,
      text=textdark,
      font=\tiny\sffamily
    }
  ]
    \node[minibox] (m1) at (0,0) {Extrakce};
    \node[minibox] (m2) at (2.2cm,0) {Chunking};
    \node[minibox] (m3) at (4.4cm,0) {KG};
    \node[minibox] (m4) at (6.6cm,0) {Multi-agent};
    \node[minibox] (m5) at (8.8cm,0) {Tools};

    \draw[->, >=stealth, color=inactivegray] (m1) -- (m2);
    \draw[->, >=stealth, color=inactivegray] (m2) -- (m3);
    \draw[->, >=stealth, color=inactivegray] (m3) -- (m4);
    \draw[->, >=stealth, color=inactivegray] (m4) -- (m5);
  \end{tikzpicture}

  \vskip1.5cm
  {\small\color{inactivegray}\insertauthor{} · \insertdate}
  \vfill
\end{frame}

% ============================================================================
% SEKCE 1: EXTRAKCE
% ============================================================================
\setpipelinestep{1}

\begin{frame}{Extrakce dokumentů}
  \begin{itemize}
    \item PDF dokumenty → strukturovaný text
    \item Gemini API pro komplexní dokumenty
    \item Zachování hierarchie (sekce, podsekce)
    \item Metadata extrakce (autor, datum, verze)
  \end{itemize}

  \vfill
  % Placeholder pro diagram/obrázek
  \centering
  \textcolor{inactivegray}{\textit{[Diagram extrakčního procesu]}}
\end{frame}

\begin{frame}{Extrakce -- Technické detaily}
  \begin{columns}[T]
    \begin{column}{0.5\textwidth}
      \textbf{Vstup}
      \begin{itemize}
        \item PDF, DOCX, TXT
        \item Scany (OCR)
        \item Tabulky, obrázky
      \end{itemize}
    \end{column}
    \begin{column}{0.5\textwidth}
      \textbf{Výstup}
      \begin{itemize}
        \item Strukturovaný JSON
        \item Hierarchie sekcí
        \item Čistý text
      \end{itemize}
    \end{column}
  \end{columns}
\end{frame}

% ============================================================================
% SEKCE 2: CHUNKING
% ============================================================================
\setpipelinestep{2}

\begin{frame}{Chunking}
  \begin{itemize}
    \item Token-aware chunking (512 tokenů)
    \item Zachování kontextu mezi chunky
    \item Summary-Augmented Chunking (SAC)
    \item Multi-layer embeddings (dokument/sekce/chunk)
  \end{itemize}

  \vfill
  \centering
  \textcolor{inactivegray}{\textit{[Vizualizace chunking procesu]}}
\end{frame}

\begin{frame}{Chunking -- SAC metoda}
  \textbf{Summary-Augmented Chunking}

  \vskip0.5cm

  \begin{enumerate}
    \item Generování kontextového shrnutí pro každý chunk
    \item Prepend shrnutí během embeddingu
    \item Strip shrnutí během retrieval
  \end{enumerate}

  \vskip0.5cm

  \textbf{Výsledek:} -58\% context drift (Anthropic, 2024)
\end{frame}

% ============================================================================
% SEKCE 3: KNOWLEDGE GRAPH
% ============================================================================
\setpipelinestep{3}

\begin{frame}{Knowledge Graph}
  \begin{itemize}
    \item Graphiti temporální knowledge graph
    \item Entity a relace z dokumentů
    \item PostgreSQL + Neo4j backend
    \item Kombinace s vektorovým vyhledáváním
  \end{itemize}

  \vfill
  \centering
  \textcolor{inactivegray}{\textit{[Ukázka knowledge grafu]}}
\end{frame}

\begin{frame}{Knowledge Graph -- Architektura}
  \begin{columns}[T]
    \begin{column}{0.5\textwidth}
      \textbf{Entity}
      \begin{itemize}
        \item Osoby
        \item Organizace
        \item Dokumenty
        \item Termíny
      \end{itemize}
    \end{column}
    \begin{column}{0.5\textwidth}
      \textbf{Relace}
      \begin{itemize}
        \item Citace
        \item Definice
        \item Časové vazby
        \item Hierarchie
      \end{itemize}
    \end{column}
  \end{columns}
\end{frame}

% ============================================================================
% SEKCE 4: MULTI-AGENT CHAT
% ============================================================================
\setpipelinestep{4}

\begin{frame}{Multi-agent systém}
  \begin{itemize}
    \item Orchestrátor + specializovaní agenti
    \item LangGraph workflow
    \item Autonomní rozhodování (ne hardcoded)
    \item Human-in-the-loop podpora
  \end{itemize}

  \vfill
  \centering
  \textcolor{inactivegray}{\textit{[Diagram multi-agent architektury]}}
\end{frame}

\begin{frame}{Agenti}
  \begin{columns}[T]
    \begin{column}{0.5\textwidth}
      \begin{itemize}
        \item \textbf{Extractor} -- extrakce informací
        \item \textbf{Classifier} -- klasifikace dotazů
        \item \textbf{Compliance} -- právní analýza
        \item \textbf{Risk Verifier} -- ověření rizik
      \end{itemize}
    \end{column}
    \begin{column}{0.5\textwidth}
      \begin{itemize}
        \item \textbf{Comparator} -- porovnání dokumentů
        \item \textbf{Timeline} -- časové analýzy
        \item \textbf{Summarizer} -- sumarizace
      \end{itemize}
    \end{column}
  \end{columns}
\end{frame}

% ============================================================================
% SEKCE 5: TOOLS
% ============================================================================
\setpipelinestep{5}

\begin{frame}{Tools}
  \begin{itemize}
    \item RAG vyhledávání (HyDE + Expansion Fusion)
    \item Graph search
    \item Multi-document synthesizer
    \item Timeline tool
  \end{itemize}

  \vfill
  \centering
  \textcolor{inactivegray}{\textit{[Přehled dostupných tools]}}
\end{frame}

\begin{frame}{Tools -- Retrieval pipeline}
  \textbf{HyDE + Expansion Fusion}

  \vskip0.5cm

  \begin{enumerate}
    \item \textbf{HyDE} -- Hypothetical Document Embeddings
    \item \textbf{Query Expansion} -- rozšíření dotazu
    \item \textbf{Fusion} -- kombinace výsledků
    \item \textbf{Reranking} -- přeřazení (ms-marco)
  \end{enumerate}

  \vskip0.5cm

  Váhy: HyDE 60\%, Expansion 40\%
\end{frame}

% ============================================================================
% ZÁVĚR
% ============================================================================
\setpipelinestep{0}

\begin{frame}{Shrnutí}
  \centering

  \begin{tikzpicture}[
    summarybox/.style={
      rectangle,
      rounded corners=3pt,
      minimum height=0.7cm,
      minimum width=2cm,
      fill=activeblue,
      text=white,
      font=\small\sffamily
    }
  ]
    \node[summarybox] (m1) at (0,0) {Extrakce};
    \node[summarybox] (m2) at (2.5cm,0) {Chunking};
    \node[summarybox] (m3) at (5cm,0) {KG};
    \node[summarybox] (m4) at (7.5cm,0) {Multi-agent};
    \node[summarybox] (m5) at (10cm,0) {Tools};

    \draw[->, >=stealth, color=activeblue, thick] (m1) -- (m2);
    \draw[->, >=stealth, color=activeblue, thick] (m2) -- (m3);
    \draw[->, >=stealth, color=activeblue, thick] (m3) -- (m4);
    \draw[->, >=stealth, color=activeblue, thick] (m4) -- (m5);
  \end{tikzpicture}

  \vskip1.5cm

  \begin{itemize}
    \item Production-ready RAG systém
    \item Založeno na výzkumných papers
    \item LangSmith observability
  \end{itemize}
\end{frame}

\begin{frame}[plain]
  \vfill
  \centering
  {\Huge\color{textdark} Děkuji za pozornost}

  \vskip1cm
  {\large\color{inactivegray} Otázky?}
  \vfill
\end{frame}

\end{document}
